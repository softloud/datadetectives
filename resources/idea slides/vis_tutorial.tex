\documentclass{beamer}

% beamer set up
%\documentclass{beamer}
% \documentclass[handout]{beamer}

\usefonttheme{serif}
\beamertemplatenavigationsymbolsempty % Hide grey symbols along bottom right.
\setbeamercolor*{structure}{fg= MyDarkBlue} 
\setbeamertemplate{itemize items}[circle]

\AtBeginSection[]{\frame{\frametitle{Outline}%
% \tableofcontents[currentsection, hideothersubsections]}}%
%\tableofcontents[currentsection, 
%currentsubsection, 
%subsectionstyle=show/shaded,
%sectionstyle=show/shaded,
%hideothersubsections]}}
\tableofcontents[currentsection, currentsubsection]}}
\AtBeginSubsection[]{
\frame{\frametitle{Outline}
\tableofcontents[currentsection, 
currentsubsection]
}
}

\setbeamercolor{block title}{fg = MyDarkBlue}
\setbeamercolor{block title example}{fg = MyDarkGreen}
\setbeamercolor{frametitle}{fg = gray}

\usepackage{hyperref}
\usepackage{soul} % For strikethrough \st

%%%%%%%%%%%%%%%%%%%%%%%%%%%%%%%%%%
% Worked examples

%%%%%%%%%%%%%%%%%%%%%%%%%%%%%%%%%%
% Maths and theorems
\usepackage{amsmath, amssymb, amsthm, amsfonts}
\usepackage{enumerate}

%%%%%%%%%%%%%%%%%%%%%%%%%%%%
% theorem commands
%\theoremstyle{plain}
%\newtheorem{theorem}{Theorem}[section]
%\newtheorem{proposition}[theorem]{Proposition}
%\newtheorem{lemma}[theorem]{Lemma}
%\newtheorem{lemma:trans}[theorem]{Transfer Lemma}
%\newtheorem{theorem: main}[theorem]{Main Theorem}
%\newtheorem{corollary}[theorem]{Corollary}
%
%
%\theoremstyle{definition}
%\newtheorem{example}[theorem]{Example}
%\newtheorem{note}[theorem]{Note}
%\newtheorem{definition}[theorem]{Definition}
%\newtheorem{remarks}[theorem]{Remarks}
%\newtheorem{remark}[theorem]{Remark}
%\newtheorem{claim}[theorem]{Claim}


%%%%%%%%%%%%%%%%%%%%%%%%%%%%%%%%%%
% tikz 

% Packages
\usepackage{tikz}
\usepackage{fix-cm} % Overrides tex restrictions on font size
\usetikzlibrary{arrows, calc}

% Graphs
\tikzstyle{vertex} = [shape = circle, fill = black, inner sep = 2.5]

%%%%%%%%%%%%%%%%%%%%%%%%%%%%%%%%%%
% Code

% Verbatim text
\newcommand{\ttt}[1]{\texttt{#1}}

% Dummy code
\newcommand{\code}[1]{\ttt{#1}}
\newcommand{\dummycode}[1]{\textcolor{MyRose}{\code{#1}}}
\newcommand{\codecomment}[1]{\ttt{\# #1}}

% Pipe/Magrittr
\newcommand{\pipe}{ \ttt{\%$>$\% }}

%%%%%%%%%%%%%%%%%%%%%%%%%%%%%%%%%%
% Colours
\definecolor{MyLightGreen}{HTML}{a6b173}
\definecolor{MyDarkGreen}{HTML}{4B610B}

\definecolor{MyLightBlue}{HTML}{9ea4bc}
\definecolor{MyDarkBlue}{HTML}{0B2F3A}

\definecolor{MyLightGrey}{HTML}{c7c8c9}
\definecolor{MyRose}{HTML}{b18ea9}

% Text colours
\newcommand{\defn}[1]{\textcolor{MyDarkGreen}{#1}}


%%%%%%%%%%%%%%%%%%%%%%%%%%%%%%%%%%
% Lecture commands

\newcommand{\timetomath}{
\begin{frame}
\begin{center}
\defn{Time to math}
\end{center}
\end{frame}
}

\newcommand{\exercise}[1]{
\begin{center}
\textcolor{MyDarkBlue}{\underline{Exercise}: #1}
\end{center}
}

\title{A quick a dirty guide to using ggplot2}
\subtitle{with data wrangling tips}
\author{Charles T. Gray}
\institute{La Trobe University}

\begin{document}
%%%%%%%%%%%%%%%%%%%%%%%%%%%%%%%%%
\begin{frame}
\maketitle
\end{frame}

%%%%%%%%%%%%%%%%%
\begin{frame}{Outline}
\tableofcontents
\end{frame}

%%%%%%%%%%%%%%%%%%%%%%%%%%%%%%%%%
\section{Themes and objectives}
\begin{frame}{Themes and objectives}
\begin{itemize}
\item Leave today confident with everyday ggplotting.
\item Also learn some of the symbiotic data wrangling techniques.
\item \defn{Leave no person behind} style of coding workshop.
\item Use data visualisation and wrangling to explore who survived the Titanic's sinking.
\end{itemize}
\end{frame}

%%%%%%%%%%%%%%%%%%%%%%%%%%%%%%%%%
\section{The basic structure of a ggplot}
\begin{frame}{Structure of a ggplot}
\textbf{ggplot2} is structured according to the \defn{grammar of graphics}.

\bigskip

The basic components of a \code{ggplot} are:
\begin{description}
\item[ggplot]  a \defn{mapping} of a dataset, so that R can interpret your points visually;
\item [aes] \defn{aesthetics}, providing details such as x and y axes;
\item [geom] \defn{geometry}, specify type of plot.
\end{description}
\end{frame}
%%%%%%%%%%%%%%%%%%%%%%%%%%%%%%%%%
\subsection{Getting started}
\begin{frame}[fragile]{Packages}
We begin by installing some packages using \ttt{library()}:
\begin{itemize}
\item \ttt{dplyr} for data wrangling;
\item \ttt{tidyr} for data wrangling;
\item \ttt{ggplot2} for data visualisation.
\end{itemize}

These packages require dataframes with rows as observations and columns as variables.
\end{frame}


%%%%%%%%%%%%%%%%%
\begin{frame}{Titanic data}
Today we will use survival data from the Titanic, which you may be familiar with from such `documentaries' as: 
\bigskip

\textbf{Titanic (1997)}\\\
\begin{tiny}
\emph {Directed by James Cameron. Ultra-Condensed by Samuel Stoddard (Movie-a-minute).}
\end{tiny}
\begin{small}
\textbf{Leonardo DiCaprio}\\

\qquad  Your social class is stuffy. Let's dance with the ship's rats and have fun. \\

\textbf{Kate Winslet}\\

\qquad  You have captured my heart. Let's run around the ship and giggle. \\

\emph{(The ship SINKS.)}\\

\textbf{Leonardo DiCaprio}\\

\qquad    Never let go. \\

\textbf{Kate Winslet}\\

 \qquad   I promise. \emph{(Lets go.)}

THE END
\end{small}

\end{frame}

\begin{frame}[fragile]{Titanic data} 
\begin{verbatim}
> # A quick look at the data.
> titanic.data %>% 
+   select(name, survived, age) %>% 
+   head(3)
                            name survived     age
1  Allen, Miss. Elisabeth Walton        1 29.0000
2 Allison, Master. Hudson Trevor        1  0.9167
3   Allison, Miss. Helen Loraine        0  2.0000
\end{verbatim}

\end{frame}

%%%%%%%%%%%%%%%%%
\subsection{Layers}
\begin{frame}{Layers: first layer}
The basic components of a \code{ggplot} are:
\begin{description}
\item[ggplot]  \underline{a \defn{mapping} of a dataset};
\item [aes] \defn{aesthetics}, providing details such as x and y axes;
\item [geom] \defn{geometry}, specify type of plot.
\end{description}
\end{frame}


%%%%%%%%%%%%%%%%%
\begin{frame}[fragile]{Layers: first layer}
Plots are built in layers, added using $+$. 

\begin{example}
\dummycode{some.data} \pipe \code{ggplot()} + \dummycode{layers}
\end{example}

\pause
\begin{definition}
The \defn{pipe} symbol \pipe passes the first object into the first argument of the subsequent function. Its keyboard shortcut is \ttt{control + shift + M}.
\end{definition}
\end{frame}

\begin{frame}{Layers: first layer}
\exercise{What happens if you pipe \code{titanic.data} into \code{ggplot()} without adding any layers?}

\begin{example}
\dummycode{some.data} \pipe \code{ggplot()}
\end{example}
\end{frame}

%%%%%%%%%%%%%%%%%
\subsection{Aesthetics}
\begin{frame}{Aesthetics: global aesthetics}
The basic components of a \code{ggplot} are:
\begin{description}
\item[ggplot]  a \defn{mapping} of a dataset;
\item [aes] \underline{\defn{aesthetics}, providing details such as x and y axes};
\item [geom] \defn{geometry}, specify type of plot.
\end{description}
\end{frame}

%%%%%%%%%%%%%%%%%
\begin{frame}{Aesthetics}
Aesthetics apply to the data (i.e., points and lines):
\begin{itemize}
\item $x$ and $y$ axes;
\item colour;
\item size;
\item transparancy;
\item grouping.
\end{itemize}

Aesthetics can apply globally across all layers, or just to a specific layer.
\end{frame}

%%%%%%%%%%%%%%%%%
\begin{frame}{Aesthetics: global aesthetics}
When you map your data, you may specify global aesthetics using \code{aes()} as an argument within \code{ggplot()}.

\pause

\begin{example}
\dummycode{some.data} \pipe \codecomment{Load some data} \\
\code{ggplot( \codecomment{Map the data for visualisation}\\
\quad aes(x = \dummycode{x.variable}, \codecomment{Specify x axis}\\
\qquad \quad y = \dummycode{y.variable}) \codecomment{Specify y axis}}\\
) 
\end{example}

\bigskip 
These aesthetics set the $x$ and $y$ variables for \underline{all} layers. 

\end{frame}

%%%%%%%%%%%%%%%%%
\begin{frame}{Aesthetics: global aesthetics}
\exercise{Create a first layer for \code{titanic.data} and set the $x$-axis to \code{age} and the $y$-axis to \code{fare}.
}
\begin{example}
\dummycode{some.data} \pipe \codecomment{Load some data} \\
\code{ggplot(} \codecomment{Map the data for visualisation}\\
\quad \code{aes(x = \dummycode{x.variable}}, \codecomment{Specify x axis}\\
\qquad \quad \code{ y = \dummycode{y.variable})} \codecomment{Specify y axis}\\
) 
\end{example}

\end{frame}


%%%%%%%%%%%%%%%%%%%%%%%%%%%%%%%%%
\subsection{Geometries}
%%%%%%%%%%%%%%%%%
\begin{frame}{Geometries}
The basic components of a \code{ggplot} are:
\begin{description}
\item[ggplot]  a \defn{mapping} of a dataset;
\item [aes] \defn{aesthetics}, providing details such as x and y axes;
\item [geom] \underline{\defn{geometry}, specify type of plot}.
\end{description}
\end{frame}

%%%%%%%%%%%%%%%%%
\begin{frame}{Geometries}
\code{Geom}s are types of plots. Each geometry function begins with \code{geom\_}:
\begin{description}
\item[\code{histogram}] 
\item[\code{point}] scatterplot
\item[\code{density}]
\item[\code{freqpoly}] a density curve of counts
\item[\code{smooth}] add a smoothed regression line (default loess)
\end{description}

\pause

\begin{example}
\dummycode{some.data} \pipe \codecomment{Load some data} \\
\code{ggplot(} \codecomment{Map the data for visualisation}\\
\quad \code{aes(x = \dummycode{x.variable}}, \codecomment{Specify x axis}\\
\qquad \quad \code{ y = \dummycode{y.variable})} $+$ \codecomment{Specify y axis}\\
\quad \code{geom\_point()} \codecomment{ Add a scatterplot layer}
\end{example}
\end{frame}

%%%%%%%%%%%%%%%%%
\begin{frame}{Geometries}
\exercise{Create a histogram of the ages of the passengers on the Titanic.}

\begin{example}
\dummycode{some.data} \pipe \codecomment{Load some data} \\
\code{ggplot(} \codecomment{Map the data for visualisation}\\
\quad \code{aes(x = \dummycode{x.variable}}, \codecomment{Specify x axis}\\
\qquad \quad \code{ y = \dummycode{y.variable})} $+$ \codecomment{Specify y axis}\\
\quad \code{geom\_point()} \codecomment{ Add a scatterplot layer}
\end{example}

\end{frame}
%%%%%%%%%%%%%%%%%
\begin{frame}{Local aesthetics}

You can specify aesthetics at the layer level. For example, you can modify the geom. 

\exercise{Modify your histogram with the histogram-specific aesthetic \code{binwidth} to something more meaningful.}

\begin{example}
\dummycode{some.data} \pipe \\
\code{ggplot(} \\
\quad \code{aes(x = \dummycode{x.variable}}, \\
\qquad \quad \code{ y = \dummycode{y.variable})} $+$ \\
\quad \code{geom\_\dummycode{thisgeom}(aes(\dummycode{geom.specific = my.specification}
)} \codecomment{ Add a geom layer with geom specific aesthetic.}
\end{example}
\end{frame}

%%%%%%%%%%%%%%%%%
\begin{frame}{Data wrangling for missing values}
We can use the \code{filter()} function to select the rows we're interested in for our visualisation.

\exercise{Modify your histogram to filter out the missing age rows; i.e., produce your histogram without error messages.}

\begin{example}
\dummycode{some.data} \pipe \\
\code{filter(\dummycode{my.variable} <= \dummycode{some.number} )} \pipe \\ 
\code{ggplot(} \\
\quad \code{aes(x = \dummycode{x.variable}}, \\
\qquad \quad \code{ y = \dummycode{y.variable})} $+$ \\
\quad \code{geom\_\dummycode{thisgeom}(aes(\dummycode{geom.specific = my.specification}
)} \codecomment{ Add a geom layer with geom specific aesthetic.}
\end{example}
\end{frame}

%%%%%%%%%%%%%%%%%%%%%%%%%%%%%%%%%
\section{Multiple layers}
%%%%%%%%%%%%%%%%%
\begin{frame}{Another geom layer}
We can add more layers using $+$. 

\exercise{Add a \code{freqpoly} geom layer to your histogram.}

Notice that if you have inconsistent geom-specific aesthetics your visualisation can be somewhat confusing.

\begin{example}
\dummycode{some.data} \pipe  \\
\code{filter(\dummycode{my.variable} <= \dummycode{some.number} )} \pipe \\ 
\code{ggplot(}\\
\quad \code{aes(x = \dummycode{x.variable}}, \\
\qquad \quad \code{ y = \dummycode{y.variable})} $+$ \\
\code{geom\_\dummycode{thisgeom}(aes(\dummycode{geom.specific = my.specification}
)}  +
 \code{geom\_\dummycode{anothergeom}() \codecomment{Add another geom layer}}
\end{example}
\end{frame}

%%%%%%%%%%%%%%%%%
\begin{frame}{Other layers}
Layers aren't just geoms, but can be:
\begin{itemize} 
\item Horizontal and vertical lines;
\item Titles and axis labels;
\item Colour palettes and themes.
\end{itemize}
\end{frame}

%%%%%%%%%%%%%%%%%
\begin{frame}{Other layers: main title}
\exercise{Add a title layer to your histogram with \code{ggtitle()}.}

\begin{example}
\dummycode{some.data} \pipe  \\
\code{ggplot(}\\
\quad \code{aes(x = \dummycode{x.variable}}, \\
\qquad \quad \code{ y = \dummycode{y.variable})} $+$ \\
 \code{geom\_\dummycode{some.geom}()} $+$ \\
 \code{ggtitle(``\dummycode{my awesome title}'')} \codecomment{Add a title} 
\end{example}

\end{frame}

%%%%%%%%%%%%%%%%%%%%%%%%%%%%%%%%%
\section{Facets}
%%%%%%%%%%%%%%%%%
\begin{frame}{Facets}
Facets subdivide your plot into subplots by variables. There are two ways to facet:
\begin{description}
\item[\code{wrap}] A horizontal row of plots faceted by one variable.
\item[\code{grid}] A grid of plots faceted by two variables.
\end{description}

\begin{example}[wrap]
\dummycode{some.data} \pipe  \\
\code{ggplot(}\\
\quad \code{aes(x = \dummycode{x.variable}}, \\
\qquad \quad \code{ y = \dummycode{y.variable})} $+$ \\
 \code{geom\_\dummycode{some.geom}()} $+$ \\
\code{facet\_wrap($\sim$ \dummycode{horizontal.facet.variable}) }
\end{example}
\end{frame}
%%%%%%%%%%%%%%%%%
\begin{frame}{Facets}
Facets subdivide your plot into subplots by variables. There are two ways to facet:
\begin{description}
\item[\code{wrap}] A horizontal row of plots faceted by one variable.
\item[\code{grid}] A grid of plots faceted by two variables.
\end{description}

\begin{example}[grid]
\dummycode{some.data} \pipe  \\
\code{ggplot(}\\
\quad \code{aes(x = \dummycode{x.variable}}, \\
\qquad \quad \code{ y = \dummycode{y.variable})} $+$ \\
 \code{geom\_\dummycode{some.geom}()} $+$ \\
\code{facet\_grid (\dummycode{vertical.facet.variable} $\sim$ \dummycode{horizontal.facet.variable}) }
\end{example}

\end{frame}

%%%%%%%%%%%%%%%%%%%%%%%%%%%%%%%%%
\section{Resources}
%%%%%%%%%%%%%%%%%

\begin{frame}{Resources}
\begin{itemize}
\item RStudio data-wrangling and data-visualisation cheat sheets.
\item Edward Tufte's \emph{The Visual Display of Quantitative Information} - `The Strunk and White of visualisation'
\item Hadley Wickham's \emph{Elegant Graphics for Data Analysis} 
\end{itemize}
\end{frame}

%%%%%%%%%%%%%%%%%%%%%%%%%%%%%%%%%
\section{Exercises}
\begin{frame}{Who survived the sinking of the Titanic?}

\exercise{Create a visualisation that compares how the different genders survived the sinking of the Titanic.}
\exercise{Create a visaulisation that compares how the different classes survived the sinking of the Titanic.}
\exercise{Create a visualisation that compares how the different classes and genders survived the sinking of the Titanic.}
\exercise{Create a visualisation that compares how ages and classes survived the sinking of the Titanic.}

\end{frame}

\end{document}